% Auto-generated from responses_v2.md — 2026-02-23
% Version: v2.3 (FINAL)
% Manuscript: Risk Adjustment for ADRD in Medicare Advantage and Health Care Experiences
% Journal: JAMA
% Packages: uses color (not xcolor), no enumitem/booktabs/setspace (not installed)

\documentclass[12pt]{article}
\usepackage[margin=1in]{geometry}
\usepackage{amsmath, amssymb}
\usepackage{color}
\usepackage{hyperref}
\usepackage{parskip}
\linespread{1.3}

% ── Response document custom commands ──────────────────────────────────────

\definecolor{mygray}{gray}{0.45}

\newenvironment{refereequote}{%
  \begin{quote}\itshape\color{mygray}%
}{%
  \end{quote}%
}

\newcommand{\msloc}[1]{\textit{(#1)}}

\newcommand{\refereeheader}[1]{%
  \bigskip\noindent\rule{\linewidth}{0.4pt}\\[2pt]%
  {\large\bfseries #1}\\[-4pt]%
  \rule{\linewidth}{0.4pt}\bigskip%
}

% ── Document metadata ──────────────────────────────────────────────────────

\title{Response to Reviewers\\[6pt]
  \large ``Risk Adjustment for ADRD in Medicare Advantage\\
  and Health Care Experiences''\\[4pt]
  \normalsize Submitted to \textit{JAMA}}
\author{Wei Fu, Yuting Qian, Seyed Karimi, Hamid Zarei, Xi Chen}
\date{February 23, 2026}

\begin{document}
\maketitle
\thispagestyle{empty}

% ══════════════════════════════════════════════════════════════════════════
% OPENING LETTER
% ══════════════════════════════════════════════════════════════════════════

Dear Editor and Reviewers,

Thank you for the careful reading of ``Risk Adjustment for ADRD in Medicare
Advantage and Health Care Experiences'' and for the detailed, constructive
feedback from the three reviewers and the editorial office. We have revised
the manuscript substantially in response to all comments. The main changes
are:

\begin{itemize}
  \item We have added comprehensive tests of compositional stability in the
    ADRD treatment group, using DiD and event study designs applied to all
    baseline demographic and health-status characteristics
    \msloc{new eMethods~5; new eFigures~12--13}.

  \item We have added the four underlying DiD components (pre/post
    $\times$ treatment/control means) for all main outcome variables
    \msloc{new eTables~1--2}.

  \item We have provided a detailed breakdown of the control group by
    diagnostic subcategory and have added sensitivity analyses using
    alternative control group definitions \msloc{revised Methods;
    new eFigure~7}.

  \item We have expanded the Discussion and Limitations to address the
    MCBS panel structure, the use of self-reported diagnosis, and the
    exclusion of partial-year enrollees \msloc{revised Limitations}.

  \item We have implemented all editorial changes requested by the
    editorial office, including STROBE compliance, IRB statement, software
    and version reporting, updated P-value formatting, and revised Key
    Points and Abstract formatting.
\end{itemize}

All page and section references below refer to the \emph{revised}
manuscript. We respond to each comment in detail below.

\bigskip
\noindent Sincerely,\\[6pt]
Wei Fu, Yuting Qian, Seyed Karimi, Hamid Zarei, and Xi Chen


% ══════════════════════════════════════════════════════════════════════════
% EDITORIAL OFFICE
% ══════════════════════════════════════════════════════════════════════════

\refereeheader{Response to the Editorial Office}

\noindent We thank the editorial office for the detailed formatting and
reporting requirements. We address each item in turn below.

\subsection*{Editorial Comment 1 --- Key Points: limit to 100 words}

\begin{refereequote}
Key Points: Limit to 100 words.
\end{refereequote}

\textbf{Response:} We have revised the Key Points section to fall within
the 100-word limit \msloc{revised Key Points}.

\subsection*{Editorial Comment 2 --- Key Points, Findings: begin with study type}

\begin{refereequote}
Key Points, Findings: Begin with study type (``In this cross-sectional
study,'').
\end{refereequote}

\textbf{Response:} The Findings sentence in Key Points now begins with
``In this cross-sectional study,'' \msloc{revised Key Points, Findings}.

\subsection*{Editorial Comment 3 --- Key Points, Meaning: limit to 1 sentence}

\begin{refereequote}
Key Points, Meaning: Limit to 1 sentence.
\end{refereequote}

\textbf{Response:} We have condensed the Meaning statement to a single
sentence \msloc{revised Key Points, Meaning}.

\subsection*{Editorial Comment 4 --- Analysis date for data ending $>$3 years ago}

\begin{refereequote}
For studies with data ending $>$3 years ago, add the date(s) the analysis
was performed to the Statistical Analysis section of the Methods.
\end{refereequote}

\textbf{Response:} We have added the dates during which the analyses were
performed to the Statistical Analysis section of both the Abstract and the
main text Methods \msloc{revised Abstract, Statistical Analysis; revised
Methods, Statistical Analysis}.

\subsection*{Editorial Comment 5 --- Results: report participant count and demographics first}

\begin{refereequote}
The number of participants and summary demographic information should be
reported in the first line of the Results section (Abstract and main text).
\end{refereequote}

\textbf{Response:} We have revised the opening sentence of the Results
section in both the Abstract and the main text to report the total number
of participants (N\,=\,5,353; ADRD group: N\,=\,1,629; control group:
N\,=\,3,724) and key demographic characteristics before presenting the
association estimates \msloc{revised Abstract, Results; revised Results,
Sample Characteristics}.

\subsection*{Editorial Comment 6 --- Abstract, Results: report rates before associations}

\begin{refereequote}
Abstract, Results: Report the basic number or rates being compared before
reporting the associations found.
\end{refereequote}

\textbf{Response:} We have restructured the Abstract Results paragraph to
present baseline rates for each outcome in the treatment and control groups
before stating the DiD association estimate, 95\% CI, and P~value. For
example, we now state the pre-2020 prevalence of access barriers in the
ADRD and control groups before reporting the 6.6 percentage-point DiD
estimate \msloc{revised Abstract, Results; see also new eTable~1 for full
pre/post $\times$ group means}.

\subsection*{Editorial Comment 7 --- Abstract, Conclusions: begin with study design}

\begin{refereequote}
Abstract, Conclusions: Begin with ``In this cross-sectional study
of\ldots'' and summarize findings in past tense.
\end{refereequote}

\textbf{Response:} The Abstract Conclusions now begins with ``In this
cross-sectional study of 5,353 Medicare Advantage beneficiaries\ldots''
and all verbs have been converted to past tense \msloc{revised Abstract,
Conclusions}.

\subsection*{Editorial Comment 8 --- STROBE reporting guideline}

\begin{refereequote}
Indicate in the study Methods how this report follows the STROBE reporting
guideline for cross-sectional studies.
\end{refereequote}

\textbf{Response:} We have added a sentence to the Methods section stating
that this report follows the STROBE reporting guideline for cross-sectional
studies, and we have included the STROBE checklist as a supplemental file
\msloc{revised Methods, Statistical Analysis; new Supplement, STROBE
Checklist}.

\subsection*{Editorial Comment 9 --- Ethical review / IRB statement}

\begin{refereequote}
Add a statement to the Methods section on review and approval of the study
by an IRB or ethics committee.
\end{refereequote}

\textbf{Response:} We have added an IRB/ethics statement to the Methods
section. The study uses the MCBS Public Use Files, which are de-identified
and publicly available. The Yale School of Public Health IRB determined
that use of this data set does not constitute human subjects research (see
IRB determination letter in the Supplement) \msloc{revised Methods, Data
and Sample; Supplement, IRB determination}.

\subsection*{Editorial Comment 10 --- Race and ethnicity reporting}

\begin{refereequote}
Include an explanation of who identified participant race and ethnicity and
the source of the classifications used. List racial and ethnic categories
in alphabetical order with ``other'' last.
\end{refereequote}

\textbf{Response:} We have added a sentence to the Methods (Variables
subsection) specifying that race and ethnicity in the MCBS are based on
self-report by the beneficiary or proxy respondent, following CMS
administrative classifications. Categories in Table~1 are now listed in
alphabetical order (Hispanic, Non-Hispanic Black, Non-Hispanic White,
Other) with ``Other'' last; we have added a note defining the subcategories
included in ``Other'' \msloc{revised Methods, Variables; revised Table~1
note}.

\subsection*{Editorial Comment 11 --- Statistical Analysis: software, version, and test descriptions}

\begin{refereequote}
Provide a brief description of all statistical tests used in the study and
levels of statistical significance. Include the statistical software used,
including the version and manufacturer, along with any extension packages.
\end{refereequote}

\textbf{Response:} We have added to the Statistical Analysis section:
(a)~explicit statements that all tests are two-sided with a significance
threshold of $\alpha = 0.05$; (b)~the software name, version, and
manufacturer (Stata~18.0, StataCorp, College Station, TX); and (c)~the
extension packages used \msloc{revised Methods, Statistical Analysis}.

\subsection*{Editorial Comment 12 --- P-value reporting format}

\begin{refereequote}
P values should be exact and expressed to 2 digits to the right of the
decimal point, or to 3 digits if $<$.01.
\end{refereequote}

\textbf{Response:} We have reviewed and reformatted all P~values
throughout the manuscript and supplement to conform to this convention
\msloc{revised throughout}.

\subsection*{Editorial Comment 13 --- Figure 2: omit numerical data; create eTable}

\begin{refereequote}
The numerical data in Figure~2 will be omitted per style. Please make an
eTable for the Supplement.
\end{refereequote}

\textbf{Response:} We have removed the numerical data from Figure~2.
The underlying annual means for each outcome by group across all years
(2015--2022) are now provided in eTable~1 (``Summary of Care Experiences
by Treatment and Over Time''), which also provides the four pre/post
$\times$ group means relevant to interpreting Figure~3 \msloc{revised
Figure~2; new eTable~1}.

\subsection*{Editorial Comment 14 --- Statistical graphs: provide in editable vector format}

\begin{refereequote}
Please provide graphs in an editable vector file format, such as .wmf or
.eps, or as Excel graphs.
\end{refereequote}

\textbf{Response:} We have exported all figures directly from Stata as
\texttt{.eps} files and have attached them to the submission as separate
files \msloc{Figures~1--4, .eps format}.

\subsection*{Editorial Comment 15 --- Supplement: self-contained and readable}

\begin{refereequote}
Supplemental content is published online without editing. Please be sure
all elements are readable and have all abbreviations expanded.
\end{refereequote}

\textbf{Response:} We have reviewed the entire supplement. All eTable and
eFigure titles now expand all abbreviations in full. All supplemental
elements are cited in the main text \msloc{revised Supplement throughout}.


% ══════════════════════════════════════════════════════════════════════════
% REVIEWER 1
% ══════════════════════════════════════════════════════════════════════════

\refereeheader{Response to Reviewer 1}

\noindent We thank Reviewer~1 for the thorough and constructive review.
We address each point in turn.

\subsection*{Comment 1 --- Compositional change in the ADRD sample}

\begin{refereequote}
One simple thing to do is to use the event study and DiD models to assess
whether the average characteristics of beneficiaries with ADRD relative to
other neurological conditions changed after the payment model went into
effect.
\end{refereequote}

\textbf{Response:} We thank the reviewer for this important suggestion.
We have implemented this analysis in the revised supplement as follows.
We have added eMethods~5, eFigure~12, and eFigure~13 to the Supplement,
which present comprehensive compositional stability tests.

Specifically, we re-estimated our DiD and event study models treating each
baseline demographic and health-status characteristic as the dependent
variable: age categories, sex, race/ethnicity (five groups), education,
marital status, BMI categories, IADL/ADL limitation categories, and number
of chronic conditions \msloc{eMethods~5; eFigures~12--13}.

\textbf{Key findings:} For most characteristics---including age categories,
sex, educational attainment, and racial groups (non-Hispanic White,
Black)---the DiD estimates are close to zero and statistically
indistinguishable from zero (95\%~CI overlapping zero), suggesting limited
evidence of broad demographic compositional change. The event studies in
eFigure~13 show no anticipatory jump or discontinuity from 2019 to 2020
for these characteristics, directly addressing the concern that diagnostic
changes in the year before the payment model took effect may have altered
the treatment group's composition.

We do observe statistically significant positive DiD estimates for the
married share and Hispanic share of the ADRD group, suggesting that married
and Hispanic beneficiaries became relatively more represented in the
treatment group after 2020. We discuss the direction of implied bias in
eMethods~5: the increase in Hispanic share (a group facing greater
structural care barriers) would tend to \emph{attenuate} our estimates
toward zero, making our findings conservative; the increase in married
share (which may proxy for greater informal caregiving support) could
introduce modest upward bias, though its magnitude is limited. Overall, the
absence of a broad, coherent compositional shift---and particularly the
lack of a discrete break between 2019 and 2020 across most
characteristics---reduces concern that our findings are driven by large
compositional change \msloc{eMethods~5; eFigures~12--13; revised
Limitations}.

\subsection*{Comment 2 --- Control group differences by age and race}

\begin{refereequote}
The control group differs in some important ways from the ADRD
sample---notably age composition and race. Showing how these
characteristics change over time might help address this concern.
\end{refereequote}

\textbf{Response:} The compositional stability analysis described in
response to Comment~1 directly addresses this concern: eFigure~12 and
eFigure~13 show that the age and race composition of the ADRD group did
not change differentially relative to the control group after 2020.
Pre-existing \emph{level} differences in age and race between the two
groups---visible in Table~1---do not threaten the DiD design as long as
trends are parallel; the covariate-by-year interactions in our main model
(equation~S.1) are specifically designed to absorb heterogeneous
contemporaneous shocks that might differentially affect subgroups that
differ by age or racial composition. We have added a sentence to the
Discussion cross-referencing the compositional stability results
\msloc{revised Discussion; eFigures~12--13}.

\subsection*{Comment 3 --- Control group composition detail and sensitivity}

\begin{refereequote}
What share are in the different categories included: Parkinson's Disease
(PD), stroke/brain hemorrhage, or complete/partial paralysis? Some
discussion of these issues and sensitivity of the results to use of these
different control groups would be useful.
\end{refereequote}

\textbf{Response:} We have added the following to the Methods (Data and
Sample subsection): among the 3,724 MA beneficiaries in the control group,
47 have Parkinson's disease (PD; 1.3\%), 1,039 have complete/partial
paralysis (27.9\%), and 3,071 have stroke/brain hemorrhage (82.5\%); these
categories are not mutually exclusive \msloc{revised Methods, Data and
Sample}.

We have performed sensitivity analyses using three alternative control
group definitions, reported in eFigure~7. \textbf{Dropping PD} (the
smallest subgroup): results are substantively unchanged (access:
$\beta = -0.068$, 95\%~CI $-0.113$ to $-0.022$, $P = .004$; financial
burden: $\beta = -0.092$, 95\%~CI $-0.161$ to $-0.023$, $P = .009$).
\textbf{Stroke/brain hemorrhage only} (our preferred neurological
comparator, given shared vascular pathways and longitudinal care needs):
estimates remain consistent (access: $\beta = -0.055$, 95\%~CI $-0.102$
to $-0.009$, $P = .020$; financial burden: $\beta = -0.093$, 95\%~CI
$-0.163$ to $-0.022$, $P = .010$). \textbf{Paralysis only} (smallest and
most heterogeneous subgroup): directionally consistent for access barriers
but less precisely estimated, likely reflecting reduced sample size
\msloc{eFigure~7}. We discuss in the text that stroke is our preferred
standalone control group for conceptual and empirical reasons, but PD's
small size (N\,=\,47) makes it unlikely to drive any baseline differences
or DiD estimates \msloc{revised Methods; eFigure~7}.

\subsection*{Comment 4 --- Perceived vs.\ actual access}

\begin{refereequote}
I'm less sure about improvements in perceived access without any detail on
changes in the amount and/or type of care received. This merits some
discussion in the limitations section.
\end{refereequote}

\textbf{Response:} We have expanded the Limitations section to explicitly
distinguish between \emph{perceived} and \emph{actual} access.
Improvements in self-reported access to needed care capture whether
beneficiaries experienced barriers to obtaining care they sought, but do
not directly measure changes in healthcare utilization volumes or specific
service types. Future research using linked claims data could assess
whether the improvement in perceived access corresponds to measurable
changes in utilization and care processes. We acknowledge this as a scope
limitation of the MCBS-based approach, and distinguish it from response
bias (already addressed by our proxy respondent interaction) \msloc{revised
Limitations}.

\subsection*{Minor Comment 1 --- Dual eligible status as a covariate}

\begin{refereequote}
I did not see a control for dual eligible status. Is it included?
\end{refereequote}

\textbf{Response:} Dual-eligible status is controlled for in our main
model. In the revised supplement, eFigure~3 presents results from a
specification that explicitly adds dual-eligible status as a covariate
interacted with year indicators; results are substantively unchanged,
confirming that our main findings are not confounded by differential
trends in dual-eligible representation \msloc{eFigure~3; revised Methods,
Variables, which now explicitly lists dual-eligible status as a covariate}.

\subsection*{Minor Comment 2 --- Missing reference}

\begin{refereequote}
You are missing a reference to one of the early papers, published in
\textit{JAMA Network Open}, on changes in ADRD diagnosis and the payment
reform.
\end{refereequote}

\textbf{Response:} We thank the reviewer for drawing our attention to
this paper. We have reviewed it and added it to the Introduction and
Discussion where we discuss prior evidence on how the 2020 payment model
change affected ADRD diagnosis rates \msloc{revised Introduction; revised
Discussion}.


% ══════════════════════════════════════════════════════════════════════════
% REVIEWER 2
% ══════════════════════════════════════════════════════════════════════════

\refereeheader{Response to Reviewer 2}

\noindent We thank Reviewer~2 for the careful methodological reading.
These comments prompted us to substantially revise and expand the Methods
and Limitations sections, and to clarify the MCBS data structure more
explicitly.

\subsection*{Comment 1 --- MCBS panel structure and unit of analysis}

\begin{refereequote}
The MCBS obtains a maximum of 4 years of data per person. The event based
analysis should be the primary analysis. None of this background to the
survey or the methods is presented in the paper or supplement.
\end{refereequote}

\textbf{Response:} We have added a detailed description of the MCBS
rotating panel structure to the Methods (Data and Sample subsection), which
was indeed absent in the prior version \msloc{revised Methods, Data and
Sample}. Specifically, we now state that the MCBS is a rotating panel
survey in which each beneficiary remains in the survey for up to 3--4
consecutive years; as a result, the 2015--2022 pooled sample consists of
overlapping cohorts with varying degrees of pre/post-2020 observation.

We have also clarified the panel composition of our analytical sample.
The MCBS Public Use Files (PUF) assign a new anonymized identifier to each
beneficiary each survey year; the PUF therefore does not support linking
the same individual across years, and we cannot directly enumerate the
subset of beneficiaries with both pre- and post-2020 observations. This
limitation does not compromise our DiD or event study designs. The pooled
DiD estimand compares group-level average outcomes before and after 2020
and is valid under a repeated cross-sectional structure; the event study
identifies year-specific deviations from the 2019 reference year using all
respondents observed in each year. Both designs rely on group-level
parallel trends, not within-person variation \msloc{revised Methods, Data
and Sample}. The event study analysis in Figure~4 uses all respondents in
the analytical sample, with each individual contributing observations only
in the years they are surveyed; year 2019 is the reference year and all
respondents observed in 2019 anchor that reference point.

We agree with the reviewer that the event study is of particular
importance. In the revised manuscript, we have elevated Figure~4 (event
study) to co-primary status alongside Figure~3 (DiD), with explicit
discussion of what each estimand captures: the pooled DiD estimates the
average treatment effect across all post-2020 observations pooled; the
event study traces the dynamic evolution of this effect year by year and
provides the pre-trend test \msloc{revised Results; revised Figure~3--4
captions}.

We have added explicit discussion to the Limitations section acknowledging
that because the PUF does not carry cross-year identifiers, we cannot
isolate the within-person subset that spans both periods. We note that this
is a feature of the MCBS Public Use File design and is common to all
PUF-based studies of this survey; the pooled repeated cross-sectional DiD
remains a valid and widely used estimator under these conditions
\msloc{revised Limitations}.

\subsection*{Comment 2 --- Self-reported dementia vs.\ claims-based ADRD identification}

\begin{refereequote}
The MCBS has data sets linked to Medicare claims and encounter records. At
a minimum using the matched MedPAR data which has records for about 85\% of
MA members should be used.
\end{refereequote}

\textbf{Response:} We appreciate the reviewer's suggestion and have
investigated this carefully. The MCBS \textbf{Public Use Files (PUF)},
which we use, are a de-identified, publicly available version of the MCBS
that does not contain a beneficiary linkage identifier (such as a hashed
Medicare beneficiary ID) that would permit merging with MedPAR or Medicare
administrative encounter records. The MedPAR linkage is available only in
the MCBS \textbf{Research Identifiable Files (RIF)}, which require a CMS
Data Use Agreement and a secure data environment and were not accessible
for this project.

We have added an explicit statement of this data limitation to the
Limitations section. We also note that using the PUF is standard practice
for MCBS-based studies (e.g., Lu and Liao, 2022; Wang et al., 2024) and
that our use of self-reported ADRD is consistent with prior published work
using this data source (Sch\"{u}ssler-Fiorenza Rose et al., 2016; Wang et
al., 2024). Moreover, there are substantive reasons why self-reported
diagnosis is appropriate here: (a)~the outcome variables (perceived access
barriers, financial burden, satisfaction) are also self-reported,
maintaining consistency in the level of measurement; (b)~self-reported
ADRD likely understates true prevalence, biasing our treatment group toward
milder cases and making our estimates conservative; and (c)~the
care-experience questions in the MCBS reference experiences over the prior
year, which aligns with the self-reported exposure window \msloc{revised
Methods, Data and Sample; revised Limitations}.

\bigskip
\noindent\textit{References:}

\noindent\small Lu M, Liao X. Access to care through telehealth among
U.S.\ Medicare beneficiaries in the wake of the COVID-19 pandemic.
\textit{Front Public Health.} 2022;10:946944.
\url{https://doi.org/10.3389/fpubh.2022.946944}

\noindent Sch\"{u}ssler-Fiorenza Rose SM, Xie D, Streim JE, Pan Q, Kwong
PL, Stineman MG. Identifying neuropsychiatric disorders in the Medicare
Current Beneficiary Survey: the benefits of combining health survey and
claims data. \textit{BMC Health Serv Res.} 2016;16:537.
\url{https://doi.org/10.1186/s12913-016-1774-y}

\noindent Wang N, Seale M, Chen J. Availability and use of telehealth
services among patients with ADRD enrolled in traditional Medicare
vs.\ Medicare Advantage during the COVID-19 pandemic.
\textit{Front Public Health.} 2024;12:1346293.
\url{https://doi.org/10.3389/fpubh.2024.1346293}\normalsize

\subsection*{Comment 3 --- Selection bias from excluding partial-year MA enrollees}

\begin{refereequote}
Dropping partial time members could be excluding those who would complain
about MA so much they ``vote with their feet.''
\end{refereequote}

\textbf{Response:} We have addressed this concern directly with an
existing sensitivity analysis. In eFigure~6, we expand the analytical
sample by including MA beneficiaries who were enrolled in their current MA
plan for less than one full year (i.e., the partial-year enrollees our
main analysis excluded). We continue to observe a statistically significant
reduction in access barriers in this expanded sample, though estimates for
the remaining outcomes are attenuated and less precisely estimated.

We discuss in eMethods~3 why this attenuation is expected: partial-year
enrollees experienced both MA and FFS plan environments within the same
survey reference period, creating heterogeneity in treatment exposure that
mechanically pulls estimates toward the null. The main-analysis exclusion
of partial-year enrollees is therefore not simply a sample
restriction---it ensures that reported care experiences correspond to a
stable, well-defined MA exposure window \msloc{eFigure~6; revised
eMethods~3; revised Limitations}.

We have also added to the Limitations section an explicit acknowledgment
of the reviewer's concern: if sicker beneficiaries selectively disenroll
from MA mid-year, the full-year-enrollment restriction may underrepresent
the most severely affected individuals. We discuss both the pro (cleaner
treatment exposure) and the con (potential underrepresentation) of this
restriction, and note that the eFigure~6 results suggest our qualitative
conclusions are robust to this design choice \msloc{revised Limitations}.

\subsection*{Additional Comment --- Comorbidities in negative control analyses}

\begin{refereequote}
Often patients have more than one condition at a time, so treating each
disease one at a time may miss disease severity.
\end{refereequote}

\textbf{Response:} We have added a sentence to the Methods discussion of
the negative control analyses acknowledging this. The HCC-based payment
logic operates at the level of individual conditions, and our design
mirrors that structure. That negative control coefficients are near zero
and statistically insignificant across multiple distinct conditions
(eFigure~9)---each with different comorbidity profiles---is reassuring: if
the main result were driven by shared comorbidity patterns, we would expect
to see significant coefficients across multiple negative controls
\msloc{revised Methods, Statistical Analysis; revised eFigure~9 notes}.


% ══════════════════════════════════════════════════════════════════════════
% REVIEWER 3
% ══════════════════════════════════════════════════════════════════════════

\refereeheader{Response to Reviewer 3}

\noindent We thank Reviewer~3 for the careful reading and the specific,
actionable suggestions for improving reporting clarity.

\subsection*{Comment 1 --- Key points: abbreviations on first use}

\begin{refereequote}
Key points: please write abbreviations upon first use.
\end{refereequote}

\textbf{Response:} ``ADRD'' is now spelled out as ``Alzheimer's Disease
and Related Dementias (ADRD)'' and ``MA'' as ``Medicare Advantage (MA)''
on first mention in the Key Points section \msloc{revised Key Points}.

\subsection*{Comment 2 --- Abstract: clarify exposure and DiD design}

\begin{refereequote}
The difference-in-difference design and composition of the control group
is not clear in the abstract.
\end{refereequote}

\textbf{Response:} We have revised the Abstract (Design, Setting, and
Participants and Exposures) to explicitly state that the exposure is the
reinstatement of the ADRD HCC in the MA risk adjustment model in 2020
(comparing pre-2020 to 2020--2022), that we use a difference-in-differences
design, and that the control group consists of MA enrollees with
stroke/brain hemorrhage, complete/partial paralysis, or Parkinson's
disease---conditions already included in the risk adjustment model before
2020 \msloc{revised Abstract, Design; revised Abstract, Exposures}.

\subsection*{Comment 3 --- Abstract: state number of ADRD beneficiaries}

\begin{refereequote}
Abstract: please state the number of beneficiaries included in analysis
who had a diagnosis of ADRD.
\end{refereequote}

\textbf{Response:} We have added the ADRD subsample size (N\,=\,1,629)
to the Abstract alongside the total sample size (N\,=\,5,353)
\msloc{revised Abstract, Design}.

\subsection*{Comment 4 --- Introduction: traditional Medicare vs.\ fee-for-service}

\begin{refereequote}
Is it more accurate to describe traditional Medicare as Medicare fee for
service?
\end{refereequote}

\textbf{Response:} We have replaced ``traditional Medicare'' with
``Medicare fee-for-service (FFS)'' on first use in the Introduction, and
used ``FFS Medicare'' consistently thereafter \msloc{revised Introduction}.

\subsection*{Comment 5 --- Is the MCBS a random sample?}

\begin{refereequote}
Is your data resource from a random sample of Medicare beneficiaries?
\end{refereequote}

\textbf{Response:} We have added a sentence to the Methods clarifying
that the MCBS uses a stratified, multistage probability sample design and
is nationally representative of the Medicare population. All analyses use
the survey-provided sampling weights \msloc{revised Methods, Data and
Sample}.

\subsection*{Comment 6 --- Interrupted time series and year main effects in event study}

\begin{refereequote}
Year main effects appear to be missing from equation~B.3. Please clarify.
\end{refereequote}

\textbf{Response:} Year fixed effects ($\alpha_t$) are explicitly included
in equation~S.2 of the supplement and are estimated in all our analyses.
We have revised eMethods~2 to write out the full specification more
clearly, making the year fixed effects explicit in the notation
\msloc{revised Supplement, eMethods~2}.

Regarding interrupted time series (ITS): the event study framework with
a comparison group is equivalent to a two-group ITS with an interaction
term for Post $\times$ Treated, which is a stronger design than
single-group ITS because it differences out secular trends common to both
groups. We have added a sentence to the Methods noting this equivalence
\msloc{revised Methods, Statistical Analysis}.

\subsection*{Comment 7 --- Characterize all sensitivity analysis participants}

\begin{refereequote}
Participants who contribute data to the study should be fully
characterized.
\end{refereequote}

\textbf{Response:} We have added eTable~4 (``Summary Statistics for
Robustness Check using All Conditions as Control Group'') and
eTables~5--20 (``Summary Statistics for Negative Control Analyses'') to
the supplement, providing full summary statistics for all additional
participant groups \msloc{new eTables~4--20; revised Supplement}.

\subsection*{Comment 8 --- ``Placebo'' vs.\ ``negative control'' terminology}

\begin{refereequote}
These sound more like negative control groups.
\end{refereequote}

\textbf{Response:} We have revised the Methods, Results, and supplement
to use ``negative control analyses'' throughout in place of ``placebo
tests'' \msloc{revised Methods, Statistical Analysis; revised Results;
revised eMethods~3--4}.

\subsection*{Comment 9 --- Report four DiD components for all figures}

\begin{refereequote}
The findings in Figure~3 are difficult to interpret without knowing the
four values: pre-2020 among control, pre-2020 among ADRD, 2020+ among
control, and 2020+ among ADRD.
\end{refereequote}

\textbf{Response:} We have added eTable~2 (``Summary of Care Experiences
by Treatment''), which reports the unadjusted means for all four cells
(pre-2020 ADRD, pre-2020 control, post-2020 ADRD, post-2020 control) for
each of the four outcome variables. eTable~1 (``Summary of Care
Experiences by Treatment and Over Time'') provides these means broken down
by individual year (2015--2022), giving the full picture of annual trends
underlying Figure~2 and the direction of change underlying Figure~3.

We have also revised the notes to Figure~3 to state the direction of
change in each group that gives rise to the negative interaction
coefficients, and added a cross-reference from Figure~3 to eTable~2
\msloc{eTables~1--2; revised Figure~3 notes}.


% ══════════════════════════════════════════════════════════════════════════
% SUMMARY TABLE
% ══════════════════════════════════════════════════════════════════════════

\refereeheader{Summary: Resolution of All Analysis Items}

\noindent The table below documents how each item requiring empirical
analysis has been resolved in this revision.

\bigskip

\noindent
\begin{tabular}{|c|l|p{5.5cm}|p{5cm}|}
\hline
\textbf{\#} & \textbf{Source} & \textbf{Item} & \textbf{Resolution} \\
\hline
AN1 & R1.1 & DiD/event study on baseline characteristics &
  $\checkmark$ Done --- eMethods~5; eFigures~12--13 \\
\hline
AN2 & R1.3 & N by diagnostic subcategory (PD, stroke, paralysis) &
  $\checkmark$ Done --- stated in Methods: PD=47, paralysis=1,039, stroke=3,071 \\
\hline
AN3 & R1.3 & DiD with alternative control group subsets &
  $\checkmark$ Done --- eFigure~7 (drop PD; stroke-only; paralysis-only) \\
\hline
AN4 & R2.1 & Balanced-panel N; clarify event study sample &
  $\checkmark$ Addressed --- PUF lacks cross-year person ID; within-person
  count not computable; DiD valid as repeated cross-section; explained in
  Limitations \\
\hline
AN5 & R2.2 & Assess MedPAR linkage feasibility &
  $\checkmark$ Addressed in writing --- PUF has no linkage key; RIF
  requires DUA \\
\hline
AN6 & R2.3 & Partial-year enrollee analysis &
  $\checkmark$ Done --- eFigure~6 \\
\hline
AN7 & R3.6 & Confirm year FE in equation B.3 &
  $\checkmark$ Confirmed --- $\alpha_t$ explicit in eq.~S.2; eMethods~2 revised \\
\hline
AN8 & R3.9 & Four DiD cells for all outcomes &
  $\checkmark$ Done --- eTables~1--2 \\
\hline
\end{tabular}

\end{document}
