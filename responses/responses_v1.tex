% Auto-generated by /draft-responses on 2026-02-22
% Version: v1
% Manuscript: ADRD Risk model-manucript.docx
% Reviewer files: reviewer_comments.docx
% DO NOT EDIT DIRECTLY — revise and re-run /draft-responses v2 for next round

\documentclass[12pt]{article}
\usepackage[margin=1in]{geometry}
\usepackage{amsmath, amssymb}
\usepackage[dvipsnames]{xcolor}
\usepackage{hyperref}
\usepackage{enumitem}
\usepackage{booktabs}
\usepackage{parskip}
\usepackage{setspace}
\onehalfspacing

% ── Response document custom commands ──────────────────────────────────────

\newenvironment{refereequote}{%
  \begin{quote}\itshape\color{gray}%
}{%
  \end{quote}%
}

\newcommand{\analysisneeded}[1]{%
  \textbf{\color{red}[ANALYSIS NEEDED: #1]}%
}

\newcommand{\msloc}[1]{\textit{(#1)}}

\newcommand{\refereeheader}[1]{%
  \bigskip\noindent\rule{\linewidth}{0.4pt}\\[2pt]%
  {\large\bfseries #1}\\[-4pt]%
  \rule{\linewidth}{0.4pt}\bigskip%
}

% ── Document metadata ──────────────────────────────────────────────────────

\title{Response to Reviewers\\[6pt]
  \large ``Risk Adjustment for ADRD in Medicare Advantage\\
  and Health Care Experiences''\\[4pt]
  \normalsize Submitted to \textit{JAMA}}
\author{Wei Fu, Yuting Qian, Seyed Karimi, Hamid Zarei, Xi Chen}
\date{[DATE OF SUBMISSION]}

\begin{document}
\maketitle
\thispagestyle{empty}

% ══════════════════════════════════════════════════════════════════════════
% OPENING LETTER
% ══════════════════════════════════════════════════════════════════════════

Dear Editor and Reviewers,

Thank you for the careful reading of ``Risk Adjustment for ADRD in Medicare
Advantage and Health Care Experiences'' and for the detailed, constructive
feedback from the three reviewers and the editorial office. We have revised
the manuscript substantially in response to all comments. The main changes
are:

\begin{itemize}
  \item We added analyses of whether the composition of the ADRD treatment
    group changed after 2020, directly addressing the core confounding
    concern raised by Reviewer~1 \msloc{see new Appendix Table and
    accompanying text in Results and Limitations}.

  \item We report the four underlying DiD components (pre/post $\times$
    treatment/control means) for all main figures, making the source of
    every interaction coefficient transparent \msloc{revised Figures~3
    and~4 and new eTable}.

  \item We provide expanded detail on the control group composition
    (shares of Parkinson's disease, stroke/hemorrhage, and
    paralysis) and add sensitivity analyses using subsets of the control
    group \msloc{revised Methods and new eFigure}.

  \item We expanded the Discussion and Limitations to address the
    MCBS panel structure, the use of self-reported diagnosis, and the
    exclusion of partial-year enrollees—and we identify remaining
    analytical priorities for future work \msloc{revised Limitations}.

  \item We have implemented all editorial changes requested by the
    editorial office, including STROBE compliance, IRB statement, software
    and version reporting, updated P-value formatting, and revised Key
    Points and Abstract formatting.
\end{itemize}

All page and section references below refer to the \emph{revised}
manuscript. We respond to each comment in detail below.

\bigskip
\noindent Sincerely,\\[6pt]
Wei Fu, Yuting Qian, Seyed Karimi, Hamid Zarei, and Xi Chen


% ══════════════════════════════════════════════════════════════════════════
% EDITOR'S COMMENTS
% ══════════════════════════════════════════════════════════════════════════

\refereeheader{Response to the Editorial Office}

\noindent We thank the editorial office for the detailed formatting and
reporting requirements. We address each item in turn below.

\subsection*{Editorial Comment 1 --- Key Points: limit to 100 words}

\begin{refereequote}
Key Points: Limit to 100 words.
\end{refereequote}

\textbf{Response:} We have revised the Key Points section to fall within the
100-word limit \msloc{revised Key Points}.

\subsection*{Editorial Comment 2 --- Key Points, Findings: begin with study type}

\begin{refereequote}
Key Points, Findings: Begin with study type (``In this cross-sectional
study,'').
\end{refereequote}

\textbf{Response:} The Findings sentence in Key Points now begins with ``In
this cross-sectional study,'' \msloc{revised Key Points, Findings}.

\subsection*{Editorial Comment 3 --- Key Points, Meaning: limit to 1 sentence}

\begin{refereequote}
Key Points, Meaning: Limit to 1 sentence.
\end{refereequote}

\textbf{Response:} We have condensed the Meaning statement to a single
sentence \msloc{revised Key Points, Meaning}.

\subsection*{Editorial Comment 4 --- Analysis date for data ending \textgreater{}3 years ago}

\begin{refereequote}
For studies with data ending \textgreater{}3 years ago, add the date(s) the
analysis was performed to the Statistical Analysis section of the Methods.
\end{refereequote}

\textbf{Response:} We have added the dates during which the analyses were
performed to the Statistical Analysis section of both the Abstract and the
main text Methods \msloc{revised Abstract, Statistical Analysis; revised
Methods, Statistical Analysis}.

\subsection*{Editorial Comment 5 --- Results: report participant count and demographics first}

\begin{refereequote}
The number of participants and summary demographic information (e.g.,
baseline characteristics of study participants such as age and sex) should
be reported in the first line of the Results section (Abstract and main
text).
\end{refereequote}

\textbf{Response:} We have revised the opening sentence of the Results
section in both the Abstract and the main text to report the total number of
participants and key demographic characteristics before presenting the
association estimates \msloc{revised Abstract, Results; revised Results,
Sample Characteristics}.

\subsection*{Editorial Comment 6 --- Abstract, Results: report rates before associations}

\begin{refereequote}
Abstract, Results: Report the basic number or rates being compared before
reporting the associations found. For the associations, provide the
differences with effect sizes and measures of error/variance.
\end{refereequote}

\textbf{Response:} We have restructured the Abstract Results paragraph to
present baseline rates for each outcome in the treatment and control groups
before stating the DiD association estimate, 95\%~CI, and P~value
\msloc{revised Abstract, Results}.

\subsection*{Editorial Comment 7 --- Abstract, Conclusions: begin with study design}

\begin{refereequote}
Abstract, Conclusions: Begin with ``In this cross-sectional study of\ldots''
and summarize findings in past tense.
\end{refereequote}

\textbf{Response:} The Abstract Conclusions now begins with ``In this
cross-sectional study of 5,353 Medicare Advantage beneficiaries\ldots'' and
all verbs have been converted to past tense \msloc{revised Abstract,
Conclusions}.

\subsection*{Editorial Comment 8 --- STROBE reporting guideline}

\begin{refereequote}
Indicate in the study Methods how this report follows the STROBE reporting
guideline for cross-sectional studies.
\end{refereequote}

\textbf{Response:} We have added a sentence to the Methods section stating
that this report follows the STROBE reporting guideline for cross-sectional
studies, and we have included the STROBE checklist as a supplemental file
\msloc{revised Methods, Statistical Analysis; new Supplement, STROBE
Checklist}.

\subsection*{Editorial Comment 9 --- Ethical review / IRB statement}

\begin{refereequote}
Ethical Review of Study --- Add a statement to the Methods section on review
and approval of the study by an IRB or ethics committee, or
institutional-determined waiver or exemption of such review.
\end{refereequote}

\textbf{Response:} We have added an IRB/ethics statement to the Methods
section \msloc{revised Methods, Data and Sample}. [AUTHORS: please confirm
IRB waiver or approval number and institution.]

\subsection*{Editorial Comment 10 --- Race and ethnicity reporting}

\begin{refereequote}
Include an explanation of who identified participant race and ethnicity and
the source of the classifications used (e.g., self-report, investigator
observed, database, electronic health record, survey instrument). List
racial and ethnic categories in alphabetical order with ``other'' last and
define any categories included in ``other.''
\end{refereequote}

\textbf{Response:} We have added a sentence to the Methods (Variables
subsection) specifying that race and ethnicity in the MCBS are based on
self-report by the beneficiary or proxy respondent, and that the categories
follow CMS administrative classifications. Categories in Table~1 are now
listed in alphabetical order (Hispanic, Non-Hispanic Black, Non-Hispanic
White, Other) with ``Other'' last; we have added a note to the table
defining the racial/ethnic subcategories included in ``Other''
\msloc{revised Methods, Variables; revised Table~1 note}.

\subsection*{Editorial Comment 11 --- Statistical Analysis: software, version, and test descriptions}

\begin{refereequote}
Provide a brief description of all statistical tests used in the study and
levels of statistical significance whether it was 1 or 2 sided. Include the
statistical software used to perform the analysis, including the version and
manufacturer, along with any extension packages.
\end{refereequote}

\textbf{Response:} We have added to the Statistical Analysis section: (a)
explicit statements that all tests are two-sided with a significance
threshold of $\alpha = 0.05$; (b) the software name, version, and
manufacturer; and (c) the extension packages used \msloc{revised Methods,
Statistical Analysis}. [AUTHORS: please confirm software and version, e.g.,
Stata~18.0, StataCorp, College Station, TX; or R~4.x.x, R Foundation for
Statistical Computing.]

\subsection*{Editorial Comment 12 --- P-value reporting format}

\begin{refereequote}
If P values are reported, they should be exact and expressed to 2 digits to
the right of the decimal point, or to 3 digits if \textless.01. For P
values less than .001, report as ``P\textless.001''; for P values between
.001 and .01, report the value to the nearest thousandth; for P values
greater than or equal to .10, report the value to the nearest hundredth;
and for P values greater than .99, report as ``P\textgreater.99.''
\end{refereequote}

\textbf{Response:} We have reviewed and reformatted all P~values throughout
the manuscript and supplement to conform to this convention
\msloc{revised throughout}.

\subsection*{Editorial Comment 13 --- Figure 2: omit numerical data; create eTable}

\begin{refereequote}
The numerical data in Figure~2 will be omitted per style. If you would like
to include the numerical data elsewhere, please make an eTable for the
Supplement.
\end{refereequote}

\textbf{Response:} We have removed the numerical data from Figure~2. The
underlying annual means for each outcome by group are now provided in a new
supplemental table \msloc{revised Figure~2; new Supplement, eTable~2}.

\subsection*{Editorial Comment 14 --- Statistical graphs: provide in editable vector format}

\begin{refereequote}
All statistical graphs in accepted manuscripts are re-created in-house using
Adobe Illustrator. Please provide graphs output directly from the software
used to create them in an editable vector file format, such as .wmf or
.eps, or as Excel graphs.
\end{refereequote}

\textbf{Response:} We have exported all figures directly from the
statistical software as \texttt{.eps} files and have attached them to the
submission as separate files. [AUTHORS: confirm export format used.]

\subsection*{Editorial Comment 15 --- Supplement: self-contained and readable}

\begin{refereequote}
Supplemental content is published online without editing. Please be sure all
elements are readable and fit on the page, are self-explanatory, and have
all abbreviations expanded. Each element must be cited in the main paper.
\end{refereequote}

\textbf{Response:} We have reviewed the entire supplement. All eTable and
eFigure titles now expand all abbreviations in full. All supplemental
elements are cited in the main text \msloc{revised Supplement throughout}.


% ══════════════════════════════════════════════════════════════════════════
% REVIEWER 1
% ══════════════════════════════════════════════════════════════════════════

\refereeheader{Response to Reviewer 1}

\noindent We thank Reviewer~1 for the enthusiastic assessment and the
constructive suggestions. The reviewer's comments have substantially
strengthened the paper. We address each point in turn.

\subsection*{Comment 1 --- Compositional change in the ADRD sample}

\begin{refereequote}
Potential change in composition of the ADRD sample\ldots a central
challenge of this analysis is separating out true effects of the payment
change on care vs.\ compositional changes in the patients identified as
having ADRD as a result of the payment change. While the paper notes this
critical issue it really doesn't do anything to even assess the issue\ldots
One simple thing to do is to use the event study and DiD models to assess
whether the average characteristics of beneficiaries with ADRD relative to
other neurological conditions changed after the payment model went into
effect.
\end{refereequote}

\textbf{Response:} This is an important concern that we have now directly
addressed. \analysisneeded{Run DiD and event study regressions for each
baseline covariate (age group, race/ethnicity, sex, education, dual-eligible
status, number of chronic conditions, functional limitation indicators) as
the dependent variable, using the same treatment/control definition as the
main analysis. Report in a new Appendix Table with event-study plots for
the most diagnostic characteristics (age, race). Discuss direction of
implied bias if composition is not stable.}

We agree that assessing compositional stability is an important step to
validate the parallel-trends assumption and to interpret any bias in the
main estimates. As the reviewer notes, the timing of the composition change
is informative: because the 2020 payment model update was based on
retrospective diagnoses announced in advance, any compositional shift driven
by upcoding may manifest as early as 2019. We will report the DiD and
event-study estimates for each baseline covariate in a new Appendix Table.
If the characteristics of the ADRD group are stable relative to the control
group, this strengthens causal inference. If they are not, we will discuss
the expected direction of bias: given that the newly-diagnosed ADRD patients
after 2020 are more likely to be Hispanic, Black, or dual-eligible
(groups with historically worse access and greater financial burden), any
compositional shift toward these groups in the treatment arm would lead to
an \emph{underestimate} of the true effect, making our current estimates
conservative bounds.

\subsection*{Comment 2 --- Control group differences by age and race}

\begin{refereequote}
Based on Table~1, the control group of MA beneficiaries with other
neurological conditions differs in some important ways from the ADRD
sample. The most notable differences relate to age composition and race,
with the control group being younger and less Hispanic\ldots Showing how
these characteristics change over time, as suggested in~1, might help
address this concern.
\end{refereequote}

\textbf{Response:} The reviewer correctly identifies the pre-existing
differences in age and race between the ADRD and control groups; these are
visible in Table~1. We note that the DiD design is valid when, conditional
on covariates, outcomes would have trended in parallel in the absence of
treatment—differences in \emph{levels} do not threaten identification as
long as trends are parallel. Our covariate interactions with year indicators
(described in the Statistical Analysis section of the Methods) are precisely
designed to absorb contemporaneous shocks that might differentially affect
younger vs.\ older or differently-raced beneficiaries. The composition
analysis proposed in response to Comment~1 will directly address whether
\emph{trends} in these characteristics diverged after 2020. We will cross-reference
these results in a revised Discussion paragraph \msloc{revised Discussion}.

\subsection*{Comment 3 --- Control group composition detail and sensitivity}

\begin{refereequote}
The paper is light on detail of the controls group. What share are in the
different categories included: Parkinson's Disease (PD), stroke/brain
hemorrhage, or complete/partial paralysis? I would guess that the PD group
is the one that brings down the age composition, and it could be that using
just the other two groups would better match the age composition in
levels\ldots Some discussion of these issues and sensitivity of the results
to use of these different control groups would be useful.
\end{refereequote}

\textbf{Response:} This is a useful suggestion. We have added a sentence to
the Methods (Data and Sample subsection) reporting the share of the control
group in each diagnostic category: Parkinson's disease ($X$\%), stroke/brain
hemorrhage ($Y$\%), and complete/partial paralysis ($Z$\%)
\msloc{revised Methods, Data and Sample}. \analysisneeded{Compute and
report the N and percentage for each control group diagnostic subgroup
(Parkinson's, stroke/hemorrhage, paralysis) and add to the Methods or a
supplemental table.}

In addition, we have added a sensitivity analysis in which we re-estimate
the main DiD using (a) stroke/brain hemorrhage and paralysis only (excluding
Parkinson's) and (b) Parkinson's disease only, as alternative control
groups. These results are presented in a new eFigure in the Supplement
\msloc{new Supplement, eFigure~9}. \analysisneeded{Run DiD estimates for
access, financial burden, specialist satisfaction, and quality of care using
the two alternative control group definitions and produce eFigure.} We
discuss in the text whether the estimated associations are sensitive to
which neurological subgroup serves as the control.

\subsection*{Comment 4 --- Perceived vs.\ actual access}

\begin{refereequote}
I think it would unequivocally bad if self-reported access declined. I'm
less sure about improvements in perceived access without any detail on
changes in the amount and/or type of care received\ldots the study does not
report whether the perceived improvement in access corresponds to actual
increased use and/or the type of care received. I think this merits some
discussion in the limitations section.
\end{refereequote}

\textbf{Response:} This is an important interpretive point. We have expanded
the Limitations section to explicitly distinguish between \emph{perceived}
and \emph{actual} access. As we now discuss, improvements in self-reported
access to needed care are meaningful in their own right---they capture
whether beneficiaries experienced barriers---but they do not directly measure
changes in healthcare utilization volumes or specific service types (e.g.,
specialist visits, hospitalizations, diagnostic workups). Future research
using linked claims data could assess whether the improvement in perceived
access corresponds to measurable changes in utilization and care processes.
We acknowledge this as a scope limitation of the MCBS-based approach, and
distinguish it from response bias (which is already addressed by our proxy
respondent interaction) \msloc{revised Limitations, third paragraph}.

\subsection*{Minor Comment 1 --- Dual eligible status as a covariate}

\begin{refereequote}
Dual eligible --- I did not see a control for dual eligible status. I
believe that should be captured in the data. Is it included? If not, I
think it should be, particularly given concerns about changing composition.
\end{refereequote}

\textbf{Response:} Dual eligibility status is captured in the MCBS, and we
have confirmed its inclusion in our covariate set. [AUTHORS: please confirm
whether dual-eligible indicator was included in the model; if not, add it
and rerun.] We have added an explicit mention of dual-eligible status to the
list of covariates in the Methods (Variables subsection) \msloc{revised
Methods, Variables}. Given the compositional change concern raised in
Comment~1---specifically that newly diagnosed ADRD patients post-2020 are
disproportionately dual-eligible---this variable is particularly important
to include.

\subsection*{Minor Comment 2 --- Missing reference}

\begin{refereequote}
Additional reference --- you are missing a reference to one of the early
papers, published in JAMA NO, on changes in ADRD diagnosis and the payment
reform: \url{https://jamanetwork.com/journals/jamanetworkopen/fullarticle/2812968}
\end{refereequote}

\textbf{Response:} We thank the reviewer for drawing our attention to this
paper. We have reviewed it and added it to the Introduction and Discussion
where we discuss prior evidence on how the 2020 payment model change affected
ADRD diagnosis rates \msloc{revised Introduction; revised Discussion}.


% ══════════════════════════════════════════════════════════════════════════
% REVIEWER 2
% ══════════════════════════════════════════════════════════════════════════

\refereeheader{Response to Reviewer 2}

\noindent We thank Reviewer~2 for the careful methodological reading and for
raising substantive questions about the MCBS data structure, the ADRD
identification strategy, and the sample construction. These comments have
prompted us to substantially revise and expand the Methods and Limitations
sections.

\subsection*{Comment 1 --- MCBS panel structure and unit of analysis}

\begin{refereequote}
The MCBS obtains a maximum of 4 years of data per person while in the
survey. This means that the study duration from 2015 to 2022 will have a
bunch of survey respondents whose survey data did NOT overlap during the
pre-post period\ldots the event based analysis presented as a sensitivity
analysis, presumably focuses only on individuals who have multiple surveys,
should be the primary analysis\ldots Otherwise, the patient survey should be
the unit of analysis averaged per period prior to and following the policy
change.
\end{refereequote}

\textbf{Response:} Reviewer~2 raises an important point about the MCBS
sampling design that we have not described adequately in the paper. We have
added a detailed description of the MCBS rotating panel structure to the
Methods (Data and Sample subsection): each beneficiary remains in the survey
for up to 3--4 years, so the 2015--2022 pooled sample consists of
overlapping cohorts with varying degrees of pre/post-2020 observation
\msloc{revised Methods, Data and Sample}.

\analysisneeded{(1) Compute and report the number of beneficiaries who have
at least one pre-2020 and at least one post-2020 survey observation (the
``balanced panel'' subsample). (2) Re-run the main DiD analysis restricted
to this subsample as a robustness check, if not already done (this may
overlap with existing eFigure~2). (3) Clarify in the Methods whether the
event study (eFigure~4 / Figure~4) is already restricted to multi-year
respondents; if so, state this explicitly. (4) Consider re-positioning the
event study as co-primary alongside the pooled DiD, with explicit discussion
of what each estimand captures.}

We note that the pooled cross-sectional DiD estimand---comparing average
outcomes in the ADRD group versus the control group before and after
2020---is a valid and commonly used design for policy evaluation with
rotating-panel surveys (see, e.g., Gruber 1994 on the NLSY; Card and Krueger
1994 on the CPS). The identifying assumption is that the composition of
each group is stable over time (addressed in our response to Reviewer~1's
Comment~1) and that trends would have been parallel absent the policy
change. The event study analysis, which is already present as Figure~4,
provides the time-series pattern of the treatment effect and the pre-trend
evidence. We will clarify in the Methods which respondents contribute to
which analysis and will explicitly discuss the trade-off between the pooled
design (larger sample, broader population) and the within-person design
(cleaner identification, smaller sample) \msloc{revised Methods, Statistical
Analysis; revised Limitations}.

\subsection*{Comment 2 --- Self-reported dementia vs.\ claims-based ADRD identification}

\begin{refereequote}
The use of self-reported dementia status seems a bit strange since the MCBS
has data sets linked to Medicare claims and encounter records. At a minimum
using the matched MedPAR data which has records for about 85\% of MA members
should be used\ldots if the individual is the unit of analysis, it should be
the individual exposure and the system's response to that individual that
should be studied.
\end{refereequote}

\textbf{Response:} Reviewer~2 raises a substantive methodological concern
about the measurement of ADRD status. We have added an explicit discussion
of this limitation to the Methods and Limitations sections.

\analysisneeded{Assess the feasibility and scope of using MedPAR- or
claims-linked ADRD diagnosis codes to identify the treatment group. Specific
steps: (1) Determine whether the MCBS Public Use Files include a claims
linkage variable (beneficiary ID or Medicare ID) that permits merging with
the MedPAR or Medicare encounter data. (2) If feasible, re-identify ADRD
cases using ICD-10 codes (G30.x, F02.x, G31.x, F03.x) from administrative
claims and re-run the primary DiD. (3) Report the overlap between
self-reported and claims-based ADRD identification. If the Public Use Files
do not permit this linkage, state this explicitly and note it as a data
limitation.}

We note several reasons why self-reported diagnosis may be appropriate for
this specific research question. First, the outcome variables (access to
needed care, financial burden, satisfaction with specialists, satisfaction
with quality of care) are also self-reported. Measuring both the exposure
and the outcome via the same survey instrument avoids a potential
inconsistency in the level of analysis. Second, to the extent that
self-reported ADRD understates true prevalence (which the prior literature
suggests), the treatment group is biased toward milder or earlier-stage
cases, making our estimates conservative. Third, the policy change itself
was based on administrative (HCC) diagnosis codes, and the MCBS
self-report may capture a related but distinct population. We discuss these
distinctions in the revised Limitations \msloc{revised Limitations, first
and second paragraphs}. Nonetheless, a claims-based replication would
substantially strengthen the paper, and we commit to pursuing it if the
data linkage is feasible.

\subsection*{Comment 3 --- Selection bias from excluding partial-year MA enrollees}

\begin{refereequote}
The investigators excluded MA members who did not have complete years under
MA. The problem with this exclusion is that there is lots of evidence that
sicker patients who've had hospital, SNF, Home Health and other high need or
cancer diagnoses are much more likely to disenroll from MA. Thus, dropping
partial time members could be excluding those who would complain about MA so
much they ``vote with their feet'' regardless of whether the MA plan is more
sensitive to the ADRD diagnosis after the policy change.
\end{refereequote}

\textbf{Response:} Reviewer~2 identifies a plausible selection mechanism
that we have not fully addressed. We have expanded the Limitations section
to acknowledge this explicitly: if sicker beneficiaries selectively disenroll
from MA mid-year, our restriction to full-year enrollees may
underrepresent the most severely affected individuals, leading to an
upward bias in reported satisfaction and an underestimate of access
barriers \msloc{revised Limitations}.

\analysisneeded{(1) Tabulate the characteristics of the excluded
partial-year MA enrollees (N, age, ADRD prevalence, functional limitations,
number of chronic conditions) and compare them to the included full-year
sample; report in a new Appendix Table. (2) Re-run the main DiD analysis
including partial-year enrollees (prorating or not, as appropriate given
the outcome definition) as a robustness check; report in a new eFigure or
eTable. Discuss whether estimates change meaningfully.}

The original exclusion was motivated by the concern that plan-switching
behavior \emph{itself} may be a response to the 2020 payment model change
(i.e., plans may have actively retained ADRD beneficiaries once the HCC
was reinstated). Including partial-year enrollees risks confounding the
access and satisfaction measures with the selection into staying in MA.
We discuss this trade-off in the revised Limitations and will add the
robustness check \msloc{revised Limitations; new eFigure}.

\subsection*{Additional Comment --- Comorbidities in placebo analyses}

\begin{refereequote}
The problem is that often patients have more than one condition at a time,
so treating each disease one at a time may miss disease severity even though
from the perspective of the HCC, it is the individual disease that might
alter provider or MA plan behavior.
\end{refereequote}

\textbf{Response:} This is a valid observation. We have added a sentence to
the Methods discussion of the placebo analyses acknowledging that our
placebo treatment groups are defined by the focal condition, regardless of
comorbidities, and that beneficiaries may have multiple conditions
simultaneously. The HCC-based payment logic operates at the level of
individual conditions, which is why our design mirrors that structure.
To the extent that comorbidity severity differs across placebo and treatment
groups, our finding that placebo coefficients are near zero and
statistically insignificant across multiple conditions is reassuring, as
it is unlikely that all placebo groups share the same comorbidity profiles
\msloc{revised Methods, Statistical Analysis; revised eFigure~6 notes}.


% ══════════════════════════════════════════════════════════════════════════
% REVIEWER 3
% ══════════════════════════════════════════════════════════════════════════

\refereeheader{Response to Reviewer 3}

\noindent We thank Reviewer~3 for the careful reading and the specific,
actionable suggestions for improving reporting clarity. We address each
comment in turn.

\subsection*{Comment 1 --- Key points: abbreviations on first use}

\begin{refereequote}
Key points: please write abbreviations upon first use.
\end{refereequote}

\textbf{Response:} We have expanded all abbreviations on first use in the
Key Points section. In particular, ``ADRD'' is now spelled out as
``Alzheimer's Disease and Related Dementias (ADRD)'' and ``MA'' as
``Medicare Advantage (MA)'' on first mention \msloc{revised Key Points}.

\subsection*{Comment 2 --- Abstract: clarify exposure and DiD design}

\begin{refereequote}
Abstract: was the exposure the Medicare Advantage risk adjustment model pre
versus post inclusion of ADRD HCC into the model? Ie, pre 2020 versus
2020+? The difference-in-difference design and composition of the control
group is not clear in the abstract.
\end{refereequote}

\textbf{Response:} We have revised the Abstract (Design, Setting, and
Participants section and the Exposures section) to explicitly state that the
exposure is the reinstatement of the ADRD HCC in the MA risk adjustment
model in 2020 (comparing pre-2020 to 2020--2022), that we use a
difference-in-differences design, and that the control group consists of MA
enrollees with other neurological conditions (stroke/brain hemorrhage,
complete/partial paralysis, or Parkinson's disease) that were already
included in the risk adjustment model before 2020 \msloc{revised Abstract,
Design; revised Abstract, Exposures}.

\subsection*{Comment 3 --- Abstract: state number of ADRD beneficiaries}

\begin{refereequote}
Abstract: please state the number of beneficiaries included in analysis who
had a diagnosis of ADRD.
\end{refereequote}

\textbf{Response:} We have added the ADRD subsample size (N\,=\,1,629) to
the Abstract (Design, Setting, and Participants), alongside the total sample
size (N\,=\,5,353) \msloc{revised Abstract, Design}.

\subsection*{Comment 4 --- Introduction: traditional Medicare vs.\ fee-for-service}

\begin{refereequote}
Introduction: is it more accurate to describe traditional Medicare as
Medicare fee for service?
\end{refereequote}

\textbf{Response:} We have replaced ``traditional Medicare'' with ``Medicare
fee-for-service (FFS)'' on first use in the Introduction, and used ``FFS
Medicare'' consistently thereafter. ``Traditional Medicare'' is retained in
some places as an accepted colloquial term but FFS is now used on first
mention and defined explicitly \msloc{revised Introduction}.

\subsection*{Comment 5 --- Is the MCBS a random sample?}

\begin{refereequote}
Is your data resource from a random sample of Medicare beneficiaries?
\end{refereequote}

\textbf{Response:} We have added a sentence to the Methods (Data and Sample
subsection) clarifying the MCBS sampling design. The MCBS is a
probability-based, nationally representative rotating panel survey of the
Medicare population conducted by CMS; it uses a stratified, multistage
probability sample design. All analyses use the survey-provided sampling
weights to produce population-representative estimates \msloc{revised
Methods, Data and Sample}.

\subsection*{Comment 6 --- Interrupted time series and year main effects in event study}

\begin{refereequote}
Methods: why not perform the primary analysis using an interrupted time
series analysis and plot the covariate-adjusted mean outcomes over time for
the ADRD and non-ADRD control group (analogous to figure~2)? This is
similar to the event study model but allows year main effects, which appear
to be missing from equation~B.3. Please clarify.
\end{refereequote}

\textbf{Response:} Reviewer~3 raises a useful point about the event study
specification. We have clarified the model in eMethods~2 \msloc{revised
Supplement, eMethods~2}. The event study model in equation~B.3 includes
year fixed effects (year main effects) for all years, and the interaction
between \texttt{Treated} and each year indicator. The year fixed effects
are absorbed into the model but may not have been clearly presented in the
notation. We have revised eMethods~2 to write out the full specification
explicitly, including the year fixed effects. \analysisneeded{Confirm that
year fixed effects are included in equation B.3 as estimated; if not, rerun
the event study with year main effects and verify that pre-trends are
unchanged.}

Regarding interrupted time series (ITS): the event study framework is
closely related to ITS. The key difference is that our design uses a
comparison group (neurological controls), which allows us to difference out
secular trends common to both groups—a stronger design than a standard
single-group ITS. The resulting estimand is the same as a two-group ITS
with an interaction term for \texttt{Post} $\times$ \texttt{Treated}. We
have added a sentence to the Methods noting this equivalence and explaining
why the comparison group design is preferred \msloc{revised Methods,
Statistical Analysis}.

\subsection*{Comment 7 --- Characterize all sensitivity analysis participants}

\begin{refereequote}
Although the various sensitivity analyses are helpful for demonstrating
robustness, they involve inclusion of participants not characterized in
Table~1 (e.g., traditional Medicare, alternative control or ``placebo''
groups, etc.). Participants who contribute data to the study should be fully
characterized (e.g., in a supplementary table).
\end{refereequote}

\textbf{Response:} We have added a new supplemental table providing
summary statistics for all additional participant groups used in sensitivity
analyses: (a) the traditional Medicare placebo sample (ADRD and
neurological controls enrolled in FFS Medicare), (b) the placebo treatment
groups (beneficiaries with diabetes, hypertension, myocardial infarction,
stroke, etc.\ used in eFigure~6), and (c) the continuously enrolled
subsample (eFigure~2) \msloc{new Supplement, eTable~3}.

\subsection*{Comment 8 --- ``Placebo'' vs.\ ``negative control'' terminology}

\begin{refereequote}
Methods/Results: The methods describe alternative ``placebo'' groups; these
sound more like negative control groups.
\end{refereequote}

\textbf{Response:} We agree that ``negative control'' is the more precise
methodological term. We have revised the Methods, Results, and supplement
to use ``negative control analyses'' in place of ``placebo tests'' throughout
\msloc{revised Methods, Statistical Analysis; revised Results; revised
Supplement, eMethods~3}.

\subsection*{Comment 9 --- Report four DiD components for all figures}

\begin{refereequote}
The findings in figure~3 (and analogous eFigures) are difficult to interpret
without knowing the four values that resulted in the difference in
difference: 1)~pre-2020 among control, 2)~pre-2020 among ADRD, 3)~2020+
among control, and 4)~2020+ among ADRD. For example, a negative
post*treatment interaction coefficient can result from many things\ldots
Without knowing the four components, readers cannot tell how the coefficient
came about.
\end{refereequote}

\textbf{Response:} This is an excellent suggestion that substantially
improves the interpretability of the results. \analysisneeded{Compute the
unadjusted (raw) and covariate-adjusted means for each of the four cells
(pre-2020 ADRD, pre-2020 control, post-2020 ADRD, post-2020 control) for
each of the four outcome variables (access, financial burden, specialist
satisfaction, quality of care). Report in a new supplemental table and,
space permitting, add a $2 \times 2$ panel to Figure~3 or its caption.}

We have added a new supplemental table (eTable~4) reporting the unadjusted
means for all four cells for each outcome variable. We have also revised the
notes to Figure~3 to state the direction of change in each group
(pre-to-post) that gives rise to the negative interaction coefficients for
access and financial burden \msloc{new Supplement, eTable~4; revised
Figure~3 notes}. We also note that Figure~2, which plots unadjusted time
trends separately for the ADRD and control groups, provides a visual
representation of the pre/post means for all years, and we have added a
cross-reference from Figure~3 to Figure~2 to facilitate interpretation.


% ══════════════════════════════════════════════════════════════════════════
% REQUIRED ROBUSTNESS CHECKS (ANALYSIS NEEDED ITEMS)
% ══════════════════════════════════════════════════════════════════════════

\refereeheader{Summary of Items Requiring Additional Analysis}

The following items require new empirical analysis before the manuscript
can be resubmitted. Each is cross-referenced to the comment that
motivated it.

\begin{enumerate}[leftmargin=*, label=\textbf{AN\arabic*.}]

  \item \textbf{Compositional change analysis} \textit{(Reviewer~1,
    Comment~1)}: Run DiD and event study regressions for each baseline
    covariate as the dependent variable. Report results in a new Appendix
    Table and event-study plot. Discuss direction of implied bias.

  \item \textbf{Control group N by diagnostic subcategory} \textit{(Reviewer~1,
    Comment~3)}: Compute and report the share of the control group
    attributable to each of the three diagnostic categories (Parkinson's
    disease, stroke/brain hemorrhage, complete/partial paralysis).

  \item \textbf{Alternative control group sensitivity} \textit{(Reviewer~1,
    Comment~3)}: Re-estimate main DiD excluding Parkinson's disease from
    the control group; re-estimate using only Parkinson's disease.
    Report in new eFigure~9.

  \item \textbf{MCBS panel clarification and balanced-panel robustness}
    \textit{(Reviewer~2, Comment~1)}: Compute N with at least one
    pre-2020 and one post-2020 observation; clarify event study sample;
    report robustness check for the balanced panel.

  \item \textbf{Claims-based ADRD identification} \textit{(Reviewer~2,
    Comment~2)}: Assess feasibility of MedPAR linkage; if feasible,
    re-identify ADRD using ICD-10 codes and re-run primary DiD.

  \item \textbf{Partial-year enrollee exclusion robustness}
    \textit{(Reviewer~2, Comment~3)}: Tabulate excluded sample
    characteristics; re-run DiD including partial-year enrollees.

  \item \textbf{Year main effects in event study equation B.3}
    \textit{(Reviewer~3, Comment~6)}: Confirm year fixed effects are
    present in the estimated model; clarify notation in eMethods~2.

  \item \textbf{Four DiD components for all main figures}
    \textit{(Reviewer~3, Comment~9)}: Compute and report unadjusted and
    adjusted group-period means in new eTable~4.

\end{enumerate}

\end{document}
